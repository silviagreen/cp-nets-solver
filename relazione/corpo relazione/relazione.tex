\documentclass[a4paper,titlepage]{article}
\usepackage{lingmacros}
\usepackage{tree-dvips}
\usepackage[italian]{babel}
\usepackage[utf8x]{inputenc}
\usepackage{amssymb}
%\usepackage{frontespizio}

\begin{document}
\tableofcontents

\newpage

\section{Introduzione}
In questo documento verranno presentati gli argomenti trattati per lo svolgimento del progetto di Sistemi con Vincoli, realizzato dagli studenti Rango Massimiliano, Segato Silvia e Tesselli Riccardo.\\Il progetto di interesse riguarda lo sviluppo di un risolutore automatico di CP-Nets sia acicliche che cicliche che adotta varie strategie di risoluzione.\\In questo documento verrà prima introdotto il problema iniziale, successivamente verranno presentate le fasi di lavoro svolte e descritte le nostre soluzioni al problema. Infine saranno visualizzati e analizzati i risultati ottenuti.

\section{Presentazione del problema}
Una CP-Net è una struttura che rappresenta reti di preferenze condizionali. Più formalmente una CP-Net sulle variabili $V = \{X_{1},\dots, X_{n}\}$ è un grafo orientato sui nodi $X_{1},\dots, X_{n}$, ogni nodo $X_{i} \in V$ è annotato con una tabella di preferenze condizionali $CPT(X_{i})$. Ogni tabella di preferenze condizionali $CPT(X_{i})$ associa un ordine totale $\succ^{i}_{\textbf{u}}$ con ogni instanziazione \textbf{u} dei genitori di $X_{i}$ detti $Pa(X_{i}) = U$. Una CP-Net si dice aciclica se il grafo che rappresenta non contiene cicli, altrimenti è detta ciclica.\\
Nel problema dato si vogliono individuare le soluzioni ottime, se esistono, di una CP-Net aciclica o ciclica, e individuare l'ordinamento parziale delle sue soluzioni. La ricerca delle soluzioni ottime è stata affrontata con due diversi approcci.\\Nella sessione successiva verranno presentate le varie fasi di lavoro.


\section{Fasi di sviluppo}

\subsection{Generazione Casuale di CPNets}
\subsection{Studio dei paper} 

sicuri che questa serva????
\subsection{Implementazione algoritmi appresi}
\subsection{Strategie di Base}
I primi algoritmi implementati per trovare le soluzioni ottime di CP-Nets sono quelli appresi e messi in pratica a lezione.

\subsubsection{Algoritmo per CP-Nets acicliche: Sweep Forward}
L'algoritmo implementato per trovare l'unica soluzione ottima di una CP-Net aciclica è denominato Sweep Forward e consiste nel seguire le dipendenze del grafo. In particolare, consiste nell'assegnare il valore ``preferito'' ad ogni variabile nel contesto degli assegnamenti fatti per i genitori.

\subsubsection{Algoritmo per CP-Nets cicliche}
Data una CP-Net ciclica, per trovare le sue soluzioni ottime si deve ricavare un insieme di vincoli $P$, tale che le soluzioni di $P$ coincidono con l'insieme delle soluzioni ottime della CP-Net. Tale insieme $P$ lo si riesce a  trovare in tempo polinomiale.

\subsubsection{Implementazione}
Il metodo che si occupa di trovare le soluzioni ottime della CP-Net generata è il seguente
\\
\\
\centerline{\texttt{public List<Assignment> getOptimalSolution(Inference strategy, boolean findAll)}}
\\

Tale metodo adotta le due strategie descritte sopra in base alla ciclicità della CP-Net data.
Ogni soluzione ottima della CP-Net è stata implementata come assegnamento (oggetto di tipo \texttt{Assignment}); un assegnamento è una mappa $\langle$variabile, valore$\rangle$.
L'algoritmo Sweep Forward per CP-Nets acicliche è stato implementato come metodo della classe \texttt{CPNet}:\\
\\
\centerline{\texttt{public Assignment solveAcyclicCPNet(Assignment result, List<Variable> vars)}}
\\

La soluzione ottima in questo caso corrisponde alla variabile \texttt{result}.
Per ciascuna variabile indipendente, viene settato in \texttt{result} il relativo assegnamento con il valore preferito.
Dopo ciò si controllano le variabili non indipendenti, precedentemente inserite in una lista; per ciascuna di queste, si controlla se i suoi genitori hanno già un valore assegnato in \texttt{result}. Se così è, allora in base all'assegnamento dei genitori si sceglie il valore preferito per la variabile in esame e lo si usa per settare l'assegnamento di quest'ultima in \texttt{result}. Fatto ciò si toglie la variabile dalla lista e si ricomincia l'analisi con le rimanenti variabili. Qualora la variabile in esame non avesse ancora una configurazione completa per tutti i suoi genitori, la si lascia nella lista e si prosegue con la successiva.
\\
Al fine di poter attuare la strategia di risoluzione per CP-Nets cicliche, il primo passo da effettuare è la conversione da CP-Net a CSP. 
\\
I CSPs che si vengono a creare consistono di variabili, ciascuna delle quali ha come dominio \{0, 1\}, e un insieme di vincoli.
A livello implementativo, ciò corrisponde alla creazione della classe \texttt{CSP} e \texttt{Variable}, nonché dell'interfaccia \texttt{Constraint}.
 
I vincoli rispettano il seguente pattern:
\\
\centerline{ipotesi $\rightarrow$ tesi}

Tesi e ipotesi si compongono di assegnamenti. La tesi contiene l'assegnamento di un unica variabile, l'ipotesi contiene disgiunzioni di congiunzioni di assegnamenti (ad esempio, la seguente può essere una ipotesi: $( (A=1 \land B=1) \lor (A=0 \land B=0))$ ). 

Tale tipo di vincolo è stato implementato con la creazione della classe \texttt{Implies}, che implementa \texttt{Constraint}.


La classe \texttt{CPNetCSP} si occupa di creare un CSP di questo tipo, dopo aver opportunamente creato le variabili a partire dai vertici del grafo, tramite il metodo
\\
\centerline{\texttt{private List<Variable> generateVariables()}}
\\
e i vincoli a partire dalle preferenze,tramite il metodo
\\
\centerline{\texttt{private List<Implies> generateConstraints(List<Variable> variables)}}
\\

Per trovare le soluzioni del CSP così creato è stata utilizzata la classe astratta\texttt{SolvingStrategy} e la sua sottoclasse \texttt{BacktrakingStrategy}. Come dice il nome stesso, tale classe si occupa di implementare la strategia di risoluzione di un CSP attraverso la ricerca con Backtraking. L'algoritmo implementato è in grado di trovare tutte le soluzioni ottime del CSP datogli in input. Tuttavia, l'utente finale può scegliere se lasciare che l'algoritmo trovi tutte le soluzioni o si fermi alla prima soluzione.

L'utente è anche in grado di scegliere se eseguire l'algoritmo di Backtraking ``di base'', o se con Forward Checking o con propagazione di vincoli (AC3). Per permettere ciò, si è utilizzata una sottoclasse di \texttt{BacktrakingStrategy}, denominata \texttt{ImprovedBacktrakingStrategy}, e una nuova classe, \texttt{AC3Strategy}, che si occupa della propagazione di vincoli.



\subsection{Creazione dell'ordinamento parziale delle soluzioni}

Per poter ottenere l'ordinamento parziale delle soluzioni, occorre eseguire il metodo \texttt{public boolean generatePartialOrderSolution()}.
Tale metodo si occupa di creare un oggetto \texttt{PartialOrderSolutionGraph}, che rappresenta il grafo delle soluzioni. In particolare,  l'oggetto contiene una lista di oggetti \texttt{Solution}, ognuno dei quali contiene, oltre al valore della soluzione stessa che, per brevità, è una stringa (ad esempio, in una CPnet che ha, nell'ordine, A, B, C come variabili, un oggetto \texttt{Solution} ha come valore ''111", che p da intendersi come l'equivalente di {A=1, B=1, C=1}), anche la lista dei valori di quelle soluzioni che sono sotto di lei nell'ordinamento (da ora in avavnti queste verranno chiamate ''sottoliste").
Per costruire il grafo vero e proprio, viene richiamato il metodo \texttt{public void setPartialOrderSolutions()}, che si occupa di settare le sottoliste di ciascuna soluzione. In questo modo, il grafo viene memorizzato come una lista di adiacenze.
\subsection{Implementazione soluzione con local search}

\section{Risultati e statistiche}

\section{Consultivo finale}

\section{Appendice: Istruzioni d'uso}


\end{document}