\documentclass[a4paper,titlepage]{article}
\usepackage{lingmacros}
\usepackage{tree-dvips}
\usepackage[italian]{babel}
\usepackage[utf8x]{inputenc}
%\usepackage{frontespizio}

\begin{document}
\tableofcontents

\newpage

\section{Introduzione}
In questo documento verranno presentati gli argomenti trattati per lo svolgimento del progetto di Sistemi con Vincoli, realizzato dagli studenti Rango Massimiliano, Segato Silvia e Tesselli Riccardo.\\Il progetto di interesse riguarda lo sviluppo di un risolutore automatico di CP-Nets sia acicliche che cicliche che adotta varie strategie di risoluzione.\\In questo documento verrà prima introdotto il problema iniziale, successivamente verranno presentate le fasi di lavoro svolte e descritte le nostre soluzioni al problema. Infine saranno visualizzati e analizzati i risultati ottenuti.

\section{Presentazione del problema}
Una CP-Net è una struttura che rappresenta reti di preferenze condizionali. Più formalmente una CP-Net sulle variabili $V = \{X_{1},\dots, X_{n}\}$ è un grafo orientato sui nodi $X_{1},\dots, X_{n}$, ogni nodo $X_{i} \in V$ è annotato con una tabella di preferenze condizionali $CPT(X_{i})$. Ogni tabella di preferenze condizionali $CPT(X_{i})$ associa un ordine totale $\succ^{i}_{\textbf{u}}$ con ogni instanziazione \textbf{u} dei genitori di $X_{i}$ detti $Pa(X_{i}) = U$. Una CP-Net si dice aciclica se il grafo che rappresenta non contiene cicli, altrimenti è detta ciclica.\\
Nel problema dato si vogliono individuare le soluzioni ottime, se esistono, di una CP-Net aciclica o ciclica, e individuare l'ordinamento parziale delle sue soluzioni. La ricerca delle soluzioni ottime è stata affrontata con due diversi approcci.\\Nella sessione successiva verranno presentate le varie fasi di lavoro.


\section{Fasi di sviluppo}

\subsection{Generazione Casuale di CPNets}
\subsection{Studio dei paper} 

sicuri che questa serva????
\subsection{Implementazione algoritmi appresi}
\subsection{Creazione dell'ordinamento parziale delle soluzioni}
\subsection{Implementazione soluzione con local search}

\section{Risultati e statistiche}

\section{Consultivo finale}

\section{Appendice: Istruzioni d'uso}


\end{document}